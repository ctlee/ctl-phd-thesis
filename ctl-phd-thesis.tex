%!TEX program = pdflatex

% Setup the documentclass
% default options: 12pt, oneside, final
%
% fonts: 10pt, 11pt, 12pt -- are valid for UCSD dissertations.
% sides: oneside, twoside -- note that two-sided theses are not accepted
%                            by OGS.
% mode: draft, final      -- draft mode switches to single spacing,
%                            removes hyperlinks, and places a black box
%                            at every overfull hbox (check these before
%                            submission).
% chapterheads            -- Include this if you want your chapters to read:
%                              Chapter 1
%                              Title of Chapter
%                            instead of
%                              1 Title of Chapter
\documentclass[11pt, chapterheads, draft]{ucsd}

\usepackage{mathtools}
\usepackage{amsthm}
\usepackage{amssymb}
\usepackage{scrextend}
\usepackage{pslatex}
\usepackage{graphicx}
\usepackage{import}
\usepackage{xcolor}
\usepackage{xspace}

%% CAPTION
% This overrides some of the ugliness in ucsd.cls and
% allows the text to be double-spaced while letting figures,
% tables, and footnotes to be single-spaced--all OGS requirements.
% NOTE: Must appear after graphics and ams math
\makeatletter
\gdef\@ptsize{2}% 12pt documents
\let\@currsize\normalsize
\makeatother
\usepackage{setspace}
\doublespace
\usepackage[font=small, width=0.9\textwidth]{caption}

%% TIMES FONT - replacements for Computer Modern
%%   This package will replace the default font with a
%%   Times-Roman font with math support.
\usepackage[T1]{fontenc}
\usepackage{mathptmx}

%% CITATIONS
% Sets citation format
% and fixes up citations madness
\usepackage{microtype}  % avoids citations that hang into the margin

%% FOOTNOTE-MAGIC
% Enables footnotes in tables, re-referencing the same footnote multiple times.
\usepackage{footnote}
\makesavenoteenv{tabular}
\makesavenoteenv{table}

%% TABLE FORMATTING MADNESS
% Enable all sorts of fun table tricks
\usepackage{rotating}  % Enables the sideways environment (NCPW)
\usepackage{array}  % Enables "m" tabular environment http://ctan.org/pkg/array
\usepackage{booktabs}  % Enables \toprule  http://ctan.org/pkg/array
\renewcommand{\arraystretch}{1.2} % Make tables easier to read
\usepackage{multirow}

\usepackage[style=chem-acs,
backend=biber,defernumbers=true,sorting=none,
citecounter,labelnumber]{biblatex}

\usepackage{subcaption}
\usepackage{listings}

\lstset{language=C++,
  basicstyle=\ttfamily\footnotesize,
  keywordstyle=\ttfamily,
  stringstyle=\ttfamily,
  commentstyle=\ttfamily,
  morecomment=[l][]{\#}
}

\newcommand\mycommfont[1]{\footnotesize\ttfamily{#1}}

\newtheorem{theorem}{Theorem}[section]
\theoremstyle{definition}
\newtheorem{definition}{Definition}[section]
\newtheorem{lemma}[theorem]{Lemma}
% \newtheorem{prop}{Proposition}


% ALGORITHMS
\usepackage[ruled]{algorithm2e} % For algorithms
\renewcommand{\algorithmcfname}{ALGORITHM}
\SetAlFnt{\small}
\SetAlCapFnt{\small}
\SetAlCapNameFnt{\small}
\SetAlCapHSkip{0pt}
\IncMargin{-\parindent}


%% NOTE TAKING PURPOSES ONLY %%
\newcommand{\Red}[1]{{\color{red} #1}}
\newcommand{\ctl}[1]{{\xspace\color{purple} CTL: #1}}

%%%%%%%%% Helpful text macros %%%%%%%%%
\newcommand{\perm}{$P_m$\xspace}
\newcommand{\permcom}{$P_m^{\rm com}$\xspace}
\newcommand{\permexp}{$P_m^{\rm exp}$\xspace}
\newcommand{\eg}{e.g.,\xspace}
\newcommand{\ie}{i.e.,\xspace}
\newcommand{\mfpt}{$\langle t\rangle$}
\newcommand{\mfptm}{\langle t\rangle}
\newcommand{\parfrac}[2]{\frac{\partial #1}{\partial #2}}
\newcommand{\parfracc}[3]{\left(\frac{\partial #1}{\partial #2}\right)_{#3}}
\newcommand{\degrees}[0]{$^{\circ}$}
\newcommand{\degree}[0]{$^{\circ}$}
\newcommand{\surf}[1]{$#1$-surface\xspace}
\providecommand{\e}[1]{\ensuremath{\times 10^{#1}}}
\DeclareMathOperator{\Link}{Lk}
\DeclareMathOperator{\Star}{St}
\DeclareMathOperator{\Closure}{Cl}
\DeclareMathOperator{\Orient}{Orient}
\DeclareMathOperator{\Level}{Lvl}
\DeclareMathOperator{\Tangent}{Tangent}
\newcommand{\st}{\ \big| \ }
\newcommand{\mc}[1]{\mathcal{#1}}
\newcommand{\abs}[1]{ {\lvert {#1} \lvert} }
\newcommand{\largewedge}{\mbox{\Large $\wedge$}}
\newcommand{\asc}{CASC\xspace}
\newcommand{\gamer}{\texttt{GAMer}\xspace}
\newcommand{\simplex}[1]{$#1$-simplex}
\newcommand{\simplices}[1]{$#1$-simplices}
%%%%%%%%%%%%%%%%%%%%%%%%%%%%%%%%%%%%%%%

\DeclareCiteCommand{\citenum}
  {}
  {\printfield{labelnumber}}
  {}
  {}

\bibliography{journals_short,permeability,mileperm,asc}


%% HYPERLINKS
%   To create a PDF with hyperlinks, you need to include the hyperref package.
%   THIS HAS TO BE THE LAST PACKAGE INCLUDED!
%   Note that the options plainpages=false and pdfpagelabels exist
%   to fix indexing associated with having both (ii) and (2) as pages.
%   Also, all links must be black according to OGS.
%   See: http://www.tex.ac.uk/cgi-bin/texfaq2html?label=hyperdupdest
%   Note: This may not work correctly with all DVI viewers (i.e. Yap breaks).
%   NOTE: hyperref will NOT work in draft mode, as noted above.
% \usepackage[colorlinks=true, pdfstartview=FitV,
%             linkcolor=black, citecolor=black,
%             urlcolor=black, plainpages=false,
%             pdfpagelabels]{hyperref}
% \hypersetup{ pdfauthor = {Your Name Here},
%              pdftitle = {The Title of The Dissertation},
%              pdfkeywords = {Keywords for Searching},
%              pdfcreator = {pdfLaTeX with hyperref package},
%              pdfproducer = {pdfLaTeX} }
% \urlstyle{same}
% \usepackage{bookmark}
\usepackage[capitalise]{cleveref}

\begin{document}

%% FRONT MATTER
%
%  All of the front matter.
%  This includes the title, degree, dedication, vita, abstract, etc..
%  Modify the file template_frontmatter.tex to change these pages.
%!TEX root=./ctl-phd-thesis.tex


% No symbols, formulas, superscripts, or Greek letters are allowed
% in your title.
\title{Investigating Molecular Transport Properties Using Computational Modeling}

\author{Christopher Ting-Kuang Lee}
\degreeyear{2019}

% Master's Degree theses will NOT be formatted properly with this file.
\degreetitle{Doctor of Philosophy}

\field{Chemistry}
% \specialization{Anthropogeny}  % If you have a specialization, add it here

% \chair{Professor Rommie E. Amaro}
% Uncomment the next line iff you have a Co-Chair
% \cochair{Professor Cochair Semimaster}
%
% Or, uncomment the next line iff you have two equal Co-Chairs.
\cochairs{Professor Rommie E. Amaro}{Professor J. Andrew McCammon}

%  The rest of the committee members  must be alphabetized by last name.
\othermembers{
Professor Michael Gilson\\
Professor Michael Holst\\
Professor Katja Lindenberg\\
Professor Francesco Paesani\\
}
\numberofmembers{6} % |chair| + |cochair| + |othermembers|


%% START THE FRONTMATTER
%
\begin{frontmatter}

%% TITLE PAGES
%
%  This command generates the title, copyright, and signature pages.
%
\makefrontmatter

%% DEDICATION
%
%  You have three choices here:
%    1. Use the ``dedication'' environment.
%       Put in the text you want, and everything will be formated for
%       you. You'll get a perfectly respectable dedication page.
%
%
%    2. Use the ``mydedication'' environment.  If you don't like the
%       formatting of option 1, use this environment and format things
%       however you wish.
%
%    3. If you don't want a dedication, it's not required.
%
%
% \begin{dedication}

% \end{dedication}


\begin{mydedication} % You are responsible for formatting here.
  \begin{flushleft}
  \vspace{1cm}
  \textbf{To my parents Serena, and Tsengdar:} for the endless support and encouragement.\\
  \textbf{To my sister Gloria:} for challenging me to be my best.\\[1cm]
  \textbf{To Jim, and Tanya:} for sharing their home and welcoming me.\\[1cm]
  \textbf{To Linda, and Cam:} without your influence I am not sure I would be the scientist I am today.\\[3cm]
  \textbf{To Bryn:} for the courage to pursue my dreams in spite of my doubts. As a wise man once wrote, ``for keeping me human as I become a scientist.''\\[3cm]
  \end{flushleft}
\end{mydedication}



%% EPIGRAPH
%
%  The same choices that applied to the dedication apply here.
%
% \begin{epigraph} % The style file will position the text for you.
%   \emph{A careful quotation\\
%   conveys brilliance.}\\
%   ---Smarty Pants
% \end{epigraph}

% \begin{myepigraph} % You position the text yourself.
%   \vfil
%   \begin{center}
%     {\bf Think! It ain't illegal yet.}
%
% 	\emph{---George Clinton}
%   \end{center}
% \end{myepigraph}


%% SETUP THE TABLE OF CONTENTS
%
\tableofcontents
\listoffigures  % Comment if you don't have any figures
\listoftables   % Comment if you don't have any tables



%% ACKNOWLEDGEMENTS
%
%  While technically optional, you probably have someone to thank.
%  Also, a paragraph acknowledging all coauthors and publishers (if
%  you have any) is required in the acknowledgements page and as the
%  last paragraph of text at the end of each respective chapter. See
%  the OGS Formatting Manual for more information.
%
\begin{acknowledgements}
\par Rommie, Andy, Rob Swift, Jesper Sorensen
\par Ben Jagger
\par Jeff Wagner cynicism, Jamie Schiffer compassion, Alisha Caliman
\par Bryn

\par Cam, Linda, Michael Shirts

\par Steffen Lindert, Dan Mermelstein, Giulia Palermo, Clarisse Ricci, John Che, John Moody, Yinglong Miao, Robert Malmstrom, Olivia \& Levi Pierce, Lorenzo Casalino,

\par Emilia, Christian, Sophia, Siti, Pek, Sasha, Vicki Feher, Garrett Chan, They Thien Nguyen, Tavina Offutt, Adam Van Wart, Prasantha Vemu, Celia Wong, Lane Votapka,

\par \cref{chap:permeability} is a modified reprint of the material as it appears in ``Lee, C. T.; Comer, J.; Herndon, C.; Leung, N.; Pavlova, A.; Swift, R. V.; Tung, C.; Rowley, C. N.; Amaro, R. E.; Chipot, C.; et al. Simulation-Based Approaches for Determining Membrane Permeability of Small Compounds. J. Chem. Inf. Model. 2016, 56 (4), 721–733. https://doi.org/10.1021/acs.jcim.6b00022.''
The dissertation author was the primary investigator and author of this paper.

\par \cref{chap:mileperm} is a modified reprint of the material as it appears in ``Votapka, L. W.; Lee, C. T.; Amaro, R. E. Two Relations to Estimate Membrane Permeability Using Milestoning. J. Phys. Chem. B 2016, 120 (33), 8606–8616. https://doi.org/10.1021/acs.jpcb.6b02814.''
        The dissertation author was the co-primary investigator and author of this paper.
\end{acknowledgements}


%% VITA
%
%  A brief vita is required in a doctoral thesis. See the OGS
%  Formatting Manual for more information.
%
\begin{vitapage}
\begin{vita}
  \item[2010-2011] Undergraduate Research Fellow\\ Shirts Lab, Department of Chemical Engineering, University of Virginia
  \item[2011] Bachelor of Science in Chemistry\\University of Virginia
  \item[2011] Bachelors of Arts in Computer Science\\
  University of Virginia
  \item[2011] Graduate Research Fellow\\
  Columbus and Mura Labs, Department of Chemistry, University of Virginia
  \item[2013] Master of Chemistry With a Specialization in Biochemistry\\ University of Virginia
  \item[2013-2019] Graduate Research Fellow\\ Amaro and McCammon Labs, University of California, San Diego
  \item[2019] Doctor of Philosophy in Chemistry\\ University of California, San Diego
\end{vita}
\begin{publications}
  \item Malmstrom, R. D.; \textbf{Lee, C. T.}; Van Wart, A. T.; Amaro, R. E. Application of Molecular-Dynamics Based Markov State Models to Functional Proteins. J. Chem. Theory Comput. 2014, 10 (7), 2648–2657. https://doi.org/10.1021/ct5002363.
  \item Gray, C.; Price, C. W.; \textbf{Lee, C. T.}; Dewald, A. H.; Cline, M. A.; McAnany, C. E.; Columbus, L.; Mura, C. Known Structure, Unknown Function: An Inquiry-Based Undergraduate Biochemistry Laboratory Course. Biochem. Mol. Biol. Educ. 2015, 43 (4), 245–262. https://doi.org/10.1002/bmb.20873.
  \item Wagner, J. R.$^{\dagger}$; \textbf{Lee, C. T.$^{\dagger}$}; Durrant, J. D.; Malmstrom, R. D.; Feher, V. A.; Amaro, R. E. Emerging Computational Methods for the Rational Discovery of Allosteric Drugs. Chem. Rev. 2016, 116 (11), 6370–6390. https://doi.org/10.1021/acs.chemrev.5b00631.
  \item Votapka, L. W.$^{\dagger}$; \textbf{Lee, C. T.$^{\dagger}$}; Amaro, R. E. Two Relations to Estimate Membrane Permeability Using Milestoning. J. Phys. Chem. B 2016, 120 (33), 8606–8616. https://doi.org/10.1021/acs.jpcb.6b02814.
  \item \textbf{Lee, C. T.}; Comer, J.; Herndon, C.; Leung, N.; Pavlova, A.; Swift, R. V.; Tung, C.; Rowley, C. N.; Amaro, R. E.; Chipot, C.; et al. Simulation-Based Approaches for Determining Membrane Permeability of Small Compounds. J. Chem. Inf. Model. 2016, 56 (4), 721–733. https://doi.org/10.1021/acs.jcim.6b00022.
  \item \textbf{Lee, C. T.}; Amaro, R. E. Exascale Computing: A New Dawn for Computational Biology. Comput. Sci. Eng. 2018, 20 (5), 18–25. https://doi.org/10.1109/MCSE.2018.05329812.
  \item Jagger, B. R.; \textbf{Lee, C. T.}; Amaro, R. E. Quantitative Ranking of Ligand Binding Kinetics with a Multiscale Milestoning Simulation Approach. J. Phys. Chem. Lett. 2018, 9 (17), 4941–4948. https://doi.org/10.1021/acs.jpclett.8b02047.
  \item Taylor, B. C.; \textbf{Lee, C. T.}; Amaro, R. E. Structural Basis for Ligand Modulation of the CCR2 Conformational Landscape. bioRxiv 2018, 392068. https://doi.org/10.1101/392068. \textit{Submitted}
  \item \textbf{Lee, C. T.$^{\dagger}$}; Moody, J. B.$^{\dagger}$; Amaro, R. E.; McCammon, J. A.; Holst, M. The Implementation of the Colored Abstract Simplicial Complex and Its Application to Mesh Generation. 2018. \textit{Submitted}
  \item \textbf{Lee, C. T.$^{\dagger}$}; Laughlin, J. G.$^{\dagger}$, Angliviel de La Beaumelle, N.; Amaro, R. E.; McCammon, J. A.; Ramamoorthi, R.; Holst, M. J.; Rangamani, P. GAMer 2: A System for 3D Mesh Processing of Cellular Electron Micrographs. bioRxiv 2019. \textit{In Preparation}
\end{publications}
\end{vitapage}

%% ABSTRACT
%
%  Doctoral dissertation abstracts should not exceed 350 words.
%   The abstract may continue to a second page if necessary.
%
\begin{abstract}
  This dissertation will be abstract.
\end{abstract}

\end{frontmatter}


\begin{refsection}
\chapter{Introduction}
\subimport{intro/}{intro}
\printbibliography[segment=\therefsegment,heading=subbibintoc]{}
\end{refsection}

\newpage
\begin{refsection}
\chapter{Simulation-Based Approaches for Determining Membrane Permeability of Small Compounds}\label{chap:permeability}
\subimport{2015-permeability/}{perm2015}
\subimport{2015-permeability/}{perm2015supp}
\printbibliography[segment=\therefsegment,heading=subbibintoc]{}
\end{refsection}

\newpage
\begin{refsection}
\chapter{Milestoning Permeability}\label{chap:mileperm}
\subimport{2016-milestone_perm/}{mileperm}
\subimport{2016-milestone_perm/}{mileperm2016supp}
\printbibliography[segment=\therefsegment,heading=subbibintoc]{}
\end{refsection}

\newpage
\begin{refsection}
\chapter{The Implementation of the Colored Abstract Simplicial Complex and its Application to Mesh Generation}\label{chap:asc}
\subimport{2018-ASC/}{ASC2018}
\printbibliography[segment=\therefsegment,heading=subbibintoc]{}
\end{refsection}

%% APPENDIX
\appendix
\chapter{Final notes}
FOO STUFF

%% END MATTER
% \printindex %% Uncomment to display the index
%               but haven't cited in the braces.
% \bibliographystyle{alpha}  %% This is just my personal favorite style.
%                              There are many others.
%\setlength{\bibleftmargin}{0.25in}  % indent each item
%\setlength{\bibindent}{-\bibleftmargin}  % unindent the first line
%\def\baselinestretch{1.0}  % force single spacing
%\setlength{\bibitemsep}{0.16in}  % add extra space between items
% \bibliography{template}
\end{document}

