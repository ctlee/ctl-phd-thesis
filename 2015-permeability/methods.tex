\section*{Methods}

  \subsection*{System preparation}

    \par We constructed a model membrane bilayer consisting of pure dimyristoyl phosphatidylcholine (DMPC) using the CHARMM-GUI membrane builder~\cite{Wu2014a}. The membrane consists of 64 lipids per leaflet solvated by 30-\AA~water pads on either side. The number of lipids was selected to provide balance between the accuracy of the free energy calculations and the bilayer size~\cite{Hu2013}. All simulations were run using TIP3P water~\cite{Jorgensen1983} and CHARMM36 lipid parameters~\cite{Klauda2010}. Previously it has been identified that CHARMM36 is a good choice for small molecule permeability calculations~\cite{Paloncyova2014}.  Small-molecule parameters came from the CGenFF force field~\cite{Vanommeslaeghe2010}.  In the case of codeine, which is not part of the standard CGenFF distribution, compatible parameters were obtained from the Paramchem webserver~\cite{Vanommeslaeghe2012,Vanommeslaeghe2012b}.

    \par The membrane-water system was minimized and equilibrated in multiple stages. During the first stage, the membrane was constrained and water was allowed to minimize for 5,000 steps. During the second stage, the membrane was allowed to minimize for 5,000 steps and the water was constrained. During the third stage, both the water and membrane were minimized for 10,000 steps in the absence of constraints. Following minimization, 10 ns of NPT equilibration was carried out using a Lowe-Andersen thermostat~\cite{Koopman2006,Lowe1999} and Langevin barostat~\cite{Feller1995} at a temperature of 298.15 and pressure of 1 atm. The Lowe-Andersen cutoff and coupling rate was set to 2.7 \AA~and 50 ps$^{-1}$, respectively. The barostat period and decay times were set to 100 and 50 fs respectively. A 2-fs time step was used. The SETTLE algorithm~\cite{Miyamoto1992} was used to constrain all covalent bonds to hydrogen atoms. The long-range cutoff was set to 12~\AA, and short-range non-bonded and bonded interactions were calculated every time step. Long-range electrostatics were calculated using the Particle Mesh Ewald method~\cite{Darden1993} every two time steps.  All minimization, equilibration, and production dynamics were carried out using the NAMD molecular dynamics engine~\cite{Phillips2005}.  
    
Steered MD~\cite{Izrailev1997,Sotomayor2007} simulations were used to pull each permeant through the membrane at a speed of 0.7\,\AA/ns (100\,ns in total).  Coordinates of the system with the permeant at equally spaced locations over the permeation pathway were extracted from the trajectories and used as initial states for all subsequent simulations.  

  \subsection*{Umbrella Sampling}

    \par The reaction coordinate was defined as the $z$-component of the distance separating the center of mass of lipid phosphorous atoms and the heavy atoms of the permeant. For each of the permeants, a total of 71 windows spaced 1\AA~apart were used with a biasing harmonic constraint with a strength of 1.5 kcal/mol/\AA$^2$ using the collective variables module of NAMD~\cite{Henin2010}.  Each window was run for a total of at least 20 ns, totaling 1.42 $\mu$s of sampling per permeant (40 ns/window and 2.84 $\mu$s overall in the case of urea). The resulting biased probability distributions were then reweighted using the weighted histogram model (WHAM)\cite{Kumar1992,Tan2012,Chodera2007}, implemented in the g\_wham software\cite{Hub2010}, to obtain the PMF. The local diffusivity was estimated using the Hummer positional autocorrelation extension of the Woolf-Roux estimator\cite{Hummer2005,Woolf1994}
    %YW: I thought all of the diffusivity calculation was done either with generalized Langevin or Bayesian inference?
    % JC: is this not generalized Langevin?
    \begin{equation}
      D(z) = \frac{\langle\delta z^2\rangle^2}{\int_{0}^{\infty}\langle\delta z(t)\delta z(0)\rangle dt},
    \label{eq:hummer}
    \end{equation}
    where $\delta z(t) = z(t)-\langle z\rangle$. For the actual numerical calculation of $D$($z$) the integrated autocorrelation times with a sigma of $0.1$ was obtained from g\_wham. We performed a linear interpolation of the resulting PMF and $D$($z$) values at 1-\AA~intervals. The results were used to numerically integrate the inhomogeneous solubility-diffusion equation, Eq.~\ref{eq:solubility-diffusion}.

  \subsection*{Replica-exchange umbrella sampling}
    \par For Hamiltonian replica-exchange US (REUS), the same parameters as plain US were used, namely the window spacing and force constant. One additional window in bulk water was added, bringing the total to 72, in order to distribute evenly across available resources.  Window exchanges were attempted every 2\,ps. Exchange ratios varied between 20 and 30\%, which is above a minimum threshold of 10\%~\cite{Sugita1999}.  Simulations were run for 20 ns for codeine and benzoic acid and 60 ns for urea.

  \subsection*{Adaptive biasing force}
    \par %Reaction coordinate was defined as the z-axis component of the distance between the permeant and the center of the DMPC bilayer. The center of the DMPC bilayer was defined by the center of all phosphorus and nitrogen atoms. 
    For each permeant, ABF calculations were performed on nine 12-\AA~windows, centered at $z$ = -30, -22, -14, -6, 0, 6, 14, 22, 30\,\AA, respectively. In order to prevent the permeant from leaving the boundary of a window, a wall force constant of 20 kcal/mol/\AA$^2$ was used. ABF forces were collected with a bin width of  0.1\,\AA. A minimum of 500 samples were collected in each bin prior to applying the biasing force. For urea and benzoic acid, each ABF window was simulated for 100\,ns, rendering a total of 0.9 $\mu$s of sampling. For codeine, each ABF window was simulated for 300\,ns, producing a total of 2.7\,$\mu$s of sampling. The final PMFs were obtained by integrating the gradient force along the reaction coordinate via NAMD. The resulting PMFs were then symmetrized by averaging of data in the +$z$ and -$z$ directions.

  \subsection*{Multiple-walker ABF} 
%  {\bf mmm.. this part looks more like an introduction to the MW ABF method?}
% Yi and Chris C both agree to cut this part below.
 %   \par Under most circumstances, when defining a transition coordinate, we hypothesize that all the relevant slow degrees of freedom are properly captured. More questionably, we further assume that our model of the reaction coordinate embraces the fast relaxing orthogonal degrees of freedom, a necessary condition for uniform sampling and, hence, convergence of the free-energy calculation. In practice, the choice of the transition coordinate is usually guided by intuition, assuming, yet without any rigorous justification, time-scale separation. Quite unfortunately, in a host of concrete examples, where intuition would, indeed, suggest a naively low-dimensional transition coordinate, hidden free-energy barriers in the orthogonal space\cite{Zheng2008}, which are only rarely overcome in the course of the simulation, plague sampling --- a shortcoming generally referred to as quasi-nonergodicity. In its most common manifestations, diffusion along the transition coordinate appears to be thwarted by repeated attempts to traverse the virtually insurmountable  barriers that separate parallel valleys of the free-energy landscape. This symptom is characteristic of an exaggeratedly simplified transition coordinate, and, therefore, of a poor choice of the underlying coarse variables. While this shortcoming could potentially be addressed by increasing the dimensionality of the transition coordinate, multiple-replica paradigms offer a convenient framework to circumvent the problem at hand by populating the different valleys and exploring concomitantly a wide range of configurational space\cite{Minoukadeh2010,Comer2014}. This stratagem is made possible by massively parallel architectures, wherein multiple walkers along the chosen transition coordinate are handled by different computing units, exchanging information periodically. Seamless combination of importance-sampling and replica-exchange algorithms constitutes a general strategy to improve ergodic sampling\cite{Chipot2007}. 
 
 {\color{red} The underlying idea of multiple-walker ABF is to explore simultaneously different portions of the transition coordinate in multiple replicas of the simulation system.
Multiple-walker ABF with $N$ walkers consists of $N$ separate simulation systems, each including only a single permeant. The only interaction between the separate systems is indirect: the force samples accumulated by all walkers up to the current time are combined to give the biasing force experienced by each. However, as the calculation converges, the biasing force approaches a fixed profile and even this indirect interaction between the walkers becomes negligible. The role of multiple-walker ABF is to circumvent possible non-ergodicity scenarios, wherein hidden barriers in orthogonal space hamper diffusion along the transition coordinate. The primary limitation of multiple-walker ABF lies in the possibility that certain walkers become trapped in basins of the free-energy landscape, thereby diminishing the efficiency of the algorithm.
To overcome this limitation, an extension of the approach, based on
Darwinian selection, eliminates the least effective walkers, i.e., the walkers
present in already well-sampled portions of the transition coordinate,
while replicating
the most effective ones\cite{Minoukadeh2010,Comer2014c}.}
%    Multiple-walker ABF addresses the deleterious consequences of hidden free-energy barriers in orthogonal space by spawning copies, which exchange periodically information about the local force felt along the transition coordinate. While the free-energy landscape is mapped faster and with a perceptibly augmented accuracy, the proposed strategy can be further sophisticated. Darwinian selection rules are enforced in the course of the simulation, promoting through replication the walkers  that have explored large portions of the transition coordinate, and eliminating the less effective ones\cite{Minoukadeh2010,Comer2014c}. 
An implementation of the multiple-walker adaptive biasing force algorithm is available in the  popular molecular dynamics program NAMD\cite{Phillips2005}, relying on its replica-communication infrastructure. For the different free-energy calculations reported herein, eight walkers were used. Force buffers were synchronized every 5,000 molecular dynamics steps.  
    %%Simulations were run for between ???

  \subsection*{Matlab-generated PMFs and $D$($z$) profiles}
    \par The phospholipid bilayer may be regarded as juxtaposed regions with distinct properties and characteristics approximately related to the lipid headgroup and tail densities.~\cite{Marrink1994}. At the bulk water-membrane interface, the headgroup density is low. Next, progressing farther into the membrane, one encounters a strongly hydrophilic and high headgroup density region, then a strongly hydrophobic and high tail density region, and finally a low tail density region at the core of the membrane. The regions of highest PMF contribute the most to the permeability, as implied by Eq. \ref{eq:solubility-diffusion}. With these regions in mind, a rough estimate of the permeability may be determined strictly from analysis of the region of greatest PMF and interpolation to the other regions (the PMF always begins at ends at zero in the bulk water). We combined the simplified membrane model with this interpolation assumption in a Matlab program (provided in the SI). With values for the PMF and diffusivity provided at up to five points, at the center and the two interfacial regions as well as two more intermediate points, this program performs a piecewise Hermite polynomial fit to emulate a PMF and diffusivity landscape, which is then integrated according to Eq. \ref{eq:solubility-diffusion}. By varying the parameters in this script, we can determine the contribution of particular characteristics of the solute-membrane interaction to the resulting permeability.

