%Introduction (Chris L.)
%  *permeability - what is it good for?
%  *solubility-diffusion model (also mention alternative approaches like milestoning)
%  *prior results from other studies
%  *need for a more systematic comparison
%  *how we meet that need here
%  
\section*{Introduction}
\par Cells are the basic unit of life. An essential feature of cells is their encapsulating phospholipid membrane. Due to the hydrophobic effect~\cite{Tanford1979,Tanford1973}, individual phospholipids do not diffuse and tumble randomly.  Instead they form a bilayer structure with polar phosphate head groups on each side, facing the bulk solvent, with the apolar lipid tails forming a hydrophobic slab in between. This densely packed structure serves two primary purposes: (1) to contain and protect cellular machinery from the harsh external environment, and (2) to maintain ionic gradients to later harvest as energy. Lipid bilayers are an effective barrier to passive diffusion of ions and hydrophilic small molecules, such as carbohydrates, but many molecules can permeate bilayers through passive diffusion at rates that depend on bilayer composition and properties of the permeating solute. The semi-permeable nature of the membrane results in an effective ``selectivity,'' where small apolar compounds can cross the membrane at appreciable rates. In contrast to transmembrane ion channels and transporters that are carefully controlled by the cell, passive selectivity is not actively regulated, but instead arises intrinsically from the forces and fluctuations present across the membrane environment. Despite the enormous importance of passive permeability to basic cell function, a detailed mechanistic understanding of this phenomenon has yet to be achieved.
    
\par Estimation of passive permeation rates is of key importance, primarily for the delivery of candidate drugs to intracellular targets, as well as for later excretion of metabolites. For example, in 1991, $\sim$40\% of all  attrition of  drug candidates was related to adverse pharmacokinetic (PK) and bioavailability results~\cite{Kola2004}. PK attrition rates have since been reduced to about 10\%~\cite{Tsaioun2009} primarily by high-throughput experimental measures of permeability such as the parallel artificial membrane permeability assay (PAMPA)~\cite{Kansy1998,Avdeef2005} and the cell-based CaCo-2 assay~\cite{Artursson2001,VanBreemen2005}. Although these empirical methods have become a mainstay in industry, they provide little to no insight into the biophysics of membrane permeation. To gain rational insight, assay results can be used to inform linear response models such as the quantitative structure permeability relationship (QSPR)\cite{Hansch1972,Hansch1969}. Due to the nature of training models, QSPR exhibits mediocre predictive performance when compared across a broad range of experimental test sets~\cite{Stouch2003,Swift2013}. Despite advances in these technologies, neither experimental nor QSPR methods provide detailed atomistic insight into the permeation process.

\par To gain atomistic insight into the passive permeability process, physics-based methods, such as molecular dynamics (MD), have become increasingly popular. While the application of MD to passive permeability is alluring, broad adoption of the method is limited by several major outstanding challenges. First, there is much debate on the ability of current force fields to reproduce system thermodynamics and kinetics correctly~\cite{Paloncyova2014,Wang2015,Vitalini2015}. Second, the computational and human time burden is large; calculating the permeability of individual compounds can require thousands of CPU-years and months of work by an experienced researcher. Thus, in order to bring MD-based methods for passive permeability into broad practice, systematic studies addressing force field accuracy and computational efficiency of potential methods are warranted. Given the plethora of new experimental results, compound permeability is an ideal benchmark system for both computational free-energy and kinetics calculations that exist. 

\par Passive membrane permeability has traditionally been studied using the homogeneous solubility-diffusivity model~\cite{Finkelstein1968}. Later work incorporating the heterogeneous nature of lipid bilayers led to the development of the inhomogeneous solubility-diffusion model\cite{Diamond1974,Marrink1994}. The inhomogeneous solubility model is derived from the steady-state flux and assumes equilibrium across the membrane. Mathematically the potential of mean force (PMF), $W$($z$), and local diffusivity coefficient, $D$($z$), are related to the resistivity, $R$, and permeability, $P$, via the equation
    
\begin{equation}
    R = \frac{1}{P} = \int_{z_1}^{z_2} \frac{\exp[\beta W(z)]}{D(z)}dz,
    \label{eq:solubility-diffusion}
\end{equation}

where $\beta$ is the thermodynamic beta ($\beta=1/k_{B}T$), and $z$ is a collective variable describing the relative position of the solute along the transmembrane axis. The integration bounds, $z_1$ and $z_2$, are points along this axis on opposing sides of the membrane. 
% JC: this sentence is weird and doesn't seem to have a clear purpose
%While $W$($z$) and $D$($z$) can be estimated from MD simulations, note that they are functions of the position across the membrane. 
% CTL: trying to allude to the importance of enhanced sampling methods gracefully.
Both $W$($z$) and $D$($z$) can be estimated from MD simulations, provided that all $z$'s are well sampled.
% CTL: is this better?
Due to the nature of Boltzmann sampling, conventional MD is not ideal for sampling rare transition states. In order to obtain sufficient sampling of transition states, various importance sampling techniques have been developed. Some examples include umbrella sampling (US)\cite{Torrie1977}, adaptive biasing force (ABF)\cite{Rodriguez-Gomez2004,Darve2001,Darve2008a,Wei2011}, metadynamics~\cite{Laio2002}, and the Wang--Landau algorithm\cite{Wang2001}. In general, many methods can be implemented with multiple replicas (RE) or walkers (MW) to further enhance sampling\cite{Swendsen1986a}.  
\par MD simulations have long been applied to study the mechanisms and rates of permeability; for a review please refer to Ref. \citenum{Xiang2006}. Many works have used various forms of US \cite{Wilson1996,Grossfield2002,Ulander2003a,Bemporad2004,MacCallum2007,Johansson2008,MacCallum2008,Bauer2011,Tejwani2011,MacCallum2011,Paloncyova2012,Swift2013,Riahi2014,Carpenter2014,Issack2015}, adaptive biasing force \cite{Bemporad2005,Comer2014a}, metadynamics\cite{Ghaemi2012}, and kinetic master equations\cite{Parisio2013}. Passive membrane permeability can also be calculated without the use of the solubility-diffusion equation via methods such as milestoning or directional milestoning\cite{Kirmizialtin2011,Vanden-Eijnden2008,Majek2010,Bello-Rivas2015}. For example, milestoning has been used to determine the permeability of water as well as blocked tryptophan\cite{Cardenas2012,Cardenas2014}.  

\par While the aforementioned studies have examined the membrane permeability of individual solutes, there has not yet been a systematic examination of what the most effective method is for calculating $W$($z$) and $D$($z$). In particular, accurate calculations of $W$($z$) are difficult because of slowly converging orthogonal degrees of freedom, namely those related to membrane distortion and relaxation\cite{Neale2011,Neale2014}. Umbrella sampling (US) is the canonical methodological approach, in which the solute is restrained at regular intervals along $z$. The effect of the restraints are analytically removed from the probability distributions calculated from the US simulations. These distributions are then combined into a single PMF covering the complete interval of the coordinate, employing a post-treatment analysis, e.g., the weighted histogram analysis method.  Replica-exchange US improves the sampling ergodicity of US simulations by attempting periodic exchanges between neighboring replicas. Conversely, the adaptive biasing force (ABF) algorithm adjusts the biasing force over time to sample the coordinate uniformly, whereas multiple walker ABF (MW-ABF) extends this further by spawning additional concurrent simulations in poorly explored regions of the coordinate.  

\par In the present work, we systematically compare four methods: US, REUS, ABF, and MW-ABF, by calculating the permeability of urea, benzoic acid and codeine through a DMPC bilayer.
