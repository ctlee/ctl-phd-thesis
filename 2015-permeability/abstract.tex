\section{abstract}
Predicting the rate of nonfacilitated permeation of solutes across lipid bilayers is important to drug design, toxicology, and signaling. These rates can be estimated using molecular dynamics simulations combined with the inhomogeneous solubility-diffusion model, which requires calculation of the potential of mean force and position-dependent diffusivity of the solute along the transmembrane axis. In this paper, we assess the efficiency and accuracy of several methods for the calculation of the permeability of a model DMPC bilayer to urea, benzoic acid, and codeine. We compare umbrella sampling, replica exchange umbrella sampling, adaptive biasing forces, and multiple-walker adaptive biasing forces for the calculation of the transmembrane PMF. No definitive advantage for any of these methods in their ability to predict the permeability coefficient \perm~was found, provided that a sufficiently long equilibration is performed. For diffusivities, a Bayesian inference method was compared to a Generalized Langevin method, both being sensitive to chosen parameters and the slow relaxation of membrane defects.
%{\bf Although the performance for predicting log Pm was generally fair, the accuracy of these calculations could be improved by the development of more accurate force fields and methods for the calculation of the PMF and diffusion coefficient profiles.}
Agreement within 1.5 log units of the computed \perm~with experiment is found for all permeants and methods.  Remaining discrepancies can likely be attributed to limitations of the force field as well as slowly relaxing collective movements within the lipid environment.  Numerical calculations based on model profiles show that \perm~can be reliably estimated from only a few data points, leading to recommendations for calculating \perm~from simulations.

% suggestion from Chris C.:
%Although the agreement with experiment, however perfectible, is by and large reasonable, the present set of calculations brings to light aspects where improvement is desirable. While the methodology is robust and has proven appropriate for a variety of applications, sampling remains a major concern, in particular for the estimation of diffusivities, owing to the slowly relaxing collective movements within the lipid environment. The discrepancies between theory and experiment also illustrate the shortcomings of macromolecular force fields in describing small, drug-like permeants.

%YW: replacing the last sentence with something on concentrating the computing resources on a few data points? sth like: Our analysis also shows that only a few data points on the PMF and diffusivity profiles contribute significantly to Pm. For this reason, we recommend focusing computational resources on these critical points, rather than evenly spreading them over the entire transmembrane axis.
