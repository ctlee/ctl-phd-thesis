% Conclusions  (JC?)
%  *difficulties that arise in the calculations - slow membrane reorganization, solute reorientation, etc.  (here and/or maybe intro?)
%  *what method to use, when, how to prepare?
%  *where best to invest limited computational time?
%  *how much sampling is really needed to get within a certain error?  (e.g., +/- 1 log unit)
%  *take home message: can MD ever be used to reliably predict permeabilities?
%   -is there a way to streamline the process for high throughput calculations?
%

\section{Conclusions}

We have computed the membrane permeability to three compounds using a variety of simulation-based methods, namely US and ABF, along with their multiple-copy variants, REUS and MW-ABF, respectively.  These three compounds, codeine, benzoic acid, and urea, span a range of chemical properties and, most importantly, permeabilities (see Table~\ref{table:results}).  All simulation methods were able to predict the permeability within one log unit typically, except in a few cases that were off by 1.5 (see Fig.~\ref{fig:deltaP}).  Interestingly, of the four methods tested, none stood out as unequivocally better than the others.  The root-mean-square error in log units for each method was 0.821 (US), 1.23 (REUS), 1.33 (ABF), and 1.08 (MW-ABF).

Because of their similar performances, no one simulation method is recommended over another.  However, regardless of the method chosen, certain procedures can improve convergence.   It was found that simulation over the entire range, i.e., from one side of the membrane to the other, followed by symmetrization of the resulting PMF converged much faster than simulating over just half the range, i.e., from one side to the membrane center (see Fig.~S5).  Additionally, the states used to seed the windows for production sampling should be well equilibrated.  Methods such as SMD used to produce these initial states induce significant non-equilibrium perturbations to the membrane that may take hundreds of nanoseconds or more to equilibrate.  Alternatives to SMD include building the membrane {\it de novo} around the permeant at varying depths~\cite{Dorairaj2007} or introducing the permeant in a perturbative fashion at different values of $z$.  Regardless of the method, equilibration should be carried out for as long as is feasible (50-100\,ns/window at least, depending on membrane depth), particularly when the permeant's stability relies upon the spontaneous formation of membrane defects and penetration of water.

Two methods for calculating the diffusivity were explored.  The first, more commonly used Generalized Langevin method, based on restraining the solute and measuring the correlation of the system forces acting on it, was combined with the PMFs in the US and REUS; the second, the Bayesian inference method, was combined with PMFs from ABF and MW-ABF.  Both methods present a number of subtleties that prevent a straightforward calculation of $D$($z$), e.g., a proper choice of the force constant in the Generalized Langevin method or the discretization time step in Bayesian inference.  Furthermore, they are both plagued by long correlation times that are not amenable to the usual MD time scales.  Despite these caveats, they produced diffusivity profiles in rough agreement, although those from the Bayesian inference method are consistently slightly higher in the membrane than those from the Generalized Langevin method (see Fig.~\ref{fig:Dz}).

Because the solubility-diffusion model relies on two position-dependent quantities, the PMF and the diffusivity, it is natural to assume they must be calculated to determine the permeability.  However, as demonstrated in the section ``Comparison of methods and tolerance to error'', only a few data points contribute significantly to it.  Therefore, permeability can be estimated from a handful of parameters, namely the values of $W$($z$) and $D$($z$) at the membrane/water interface and at the membrane center, with the remainder interpolated from an expected smooth topology.  Going even further, extrapolating from just a single value each of the PMF and the diffusivity at the membrane center for urea (see Fig.~\ref{fig:models}A) produced a log\perm~almost identical to the experimental value.  Thus, we recommend that sampling be focused on critical regions where barriers are expected, e.g., the membrane core and/or interfacial region.

Taken together, the results show the robustness of a variety of MD-based methods in calculating membrane permeability for small molecules.  In particular, all methods {\color{red} appear to} converge on sub-$\mu$s time scales, although the determined log\perm~values are nearly all 0.5 to 1.5 log units above the experimental values.  This consistent overestimation is in agreement with a previous MD study from 2004~\cite{Bemporad2004}, despite using 10-100$\times$ longer simulations in the current study.  {\color{red} The membranes compared, DMPC in simulation with egg PC in experiment, are not identical, which may contribute to the discrepancy.}  Another possible reason could be slow membrane reorganization that occurs on a time scale even an additional 2-3 orders of magnitude greater~\cite{Neale2011}.  The lack of polarizability is another tempting possibility, given the vastly different environments experienced by the permeating molecule~\cite{Riahi2014}.
Similarly, the most commonly used force fields (including CGenFF used in this study) are primarily parameterized to reproduce phenomena obtained in aqueous environments. Thus, it may be unrealistic to expect that solute phenomena within the non-polar membrane environment is as well represented. Therefore, further exploration of the force-field effects within lipid-like or non-polar environments are warranted.  In particular, inclusion of explicit polarizability may improve accuracy, although at a computational cost of 2-6$\times$~\cite{Chowdhary2013,Wang2013}.  {\color{red} However, given the lack of convergence of the four methods to a single value for each permeant, all of which use the same force field, it cannot be known a priori to what degree, if any, changes to the force field will improve agreement with experiment.}
Additionally, the assumptions of the solubility-diffusion model, such as the reliance on a single reaction coordinate and that permeation obeys classical diffusion, may need to be challenged to obtain improved quantitative agreement with experimental measures~\cite{Orsi2010,Parisio2013,Comer2014}.

Finally, we draw attention to the potential benefit of our results to QSPR models.
%Even if the explicit-membrane permeability calculations reported here were perfectly accurate, they are still too computationally demanding to replace QSPR, taking weeks where results are needed in less than a day.  However,
We have shown that only one or two points in the PMF and diffusivity contribute significantly to the permeability.  Thus, calculations for a single permeant focusing on these critical points could be done in hours on a sufficiently fast cluster.  Furthermore, the location of these points, typically at the membrane center or at the water/membrane interface, suggest reduced systems that could represent them reasonably well, e.g., octanol for the interfacial region.
Combined with improved force fields, fast calculations on such reduced systems could augment the molecular descriptors used to design QSPR models, similar to the ``membrane-interaction QSAR'' approach pioneered by Hopfinger et al.~\cite{Kulkarni1999,Tseng2012}.



