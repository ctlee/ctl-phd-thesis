%!TEX root=./ctl-phd-thesis.tex


% No symbols, formulas, superscripts, or Greek letters are allowed
% in your title.
\title{Enabling Multiscale Modeling}

\author{Christopher Ting-Kuang Lee}
\degreeyear{2019}

% Master's Degree theses will NOT be formatted properly with this file.
\degreetitle{Doctor of Philosophy}

\field{Chemistry}
% \specialization{Anthropogeny}  % If you have a specialization, add it here

% \chair{Professor Rommie E. Amaro}
% Uncomment the next line iff you have a Co-Chair
% \cochair{Professor Cochair Semimaster}
%
% Or, uncomment the next line iff you have two equal Co-Chairs.
\cochairs{Professor Rommie E. Amaro}{Professor J. Andrew McCammon}

%  The rest of the committee members  must be alphabetized by last name.
\othermembers{
Professor Michael Gilson\\
Professor Michael Holst\\
Professor Katja Lindenberg\\
Professor Francesco Paesani
}
\numberofmembers{6} % |chair| + |cochair| + |othermembers|


%% START THE FRONTMATTER
%
\begin{frontmatter}

%% TITLE PAGES
%
%  This command generates the title, copyright, and signature pages.
%
\makefrontmatter

%% DEDICATION
%
%  You have three choices here:
%    1. Use the ``dedication'' environment.
%       Put in the text you want, and everything will be formated for
%       you. You'll get a perfectly respectable dedication page.
%
%
%    2. Use the ``mydedication'' environment.  If you don't like the
%       formatting of option 1, use this environment and format things
%       however you wish.
%
%    3. If you don't want a dedication, it's not required.
%
%
% \begin{dedication}

% \end{dedication}


\begin{mydedication} % You are responsible for formatting here.
  \begin{flushleft}
  \vspace{1cm}
  \textbf{To my parents Serena, and Tsengdar:} for the endless support and encouragement.\\
  \textbf{To my sister Gloria:} for challenging me to be my best.\\[1cm]
  \textbf{To Jim, and Tanya:} for sharing their home and welcoming me.\\[1cm]
  \textbf{To Linda, and Cam:} without your influence I am not sure I would be the scientist I am today.\\[3cm]
  \textbf{To Bryn:} for the courage to pursue my dreams in spite of my doubts. As a wise man once wrote, ``for keeping me human as I become a scientist.''
  \end{flushleft}
\end{mydedication}



%% EPIGRAPH
%
%  The same choices that applied to the dedication apply here.
%
\begin{epigraph} % The style file will position the text for you.
  \emph{So long and thanks for all the fish}\\
  ---Douglas Adams
\end{epigraph}

% \begin{myepigraph} % You position the text yourself.
%   \vfil
%   \begin{center}
%     {\bf Think! It ain't illegal yet.}
%
% 	\emph{---George Clinton}
%   \end{center}
% \end{myepigraph}


%% SETUP THE TABLE OF CONTENTS
%
\tableofcontents
\listoffigures  % Comment if you don't have any figures
\listoftables   % Comment if you don't have any tables


%% ACKNOWLEDGEMENTS
%
%  While technically optional, you probably have someone to thank.
%  Also, a paragraph acknowledging all coauthors and publishers (if
%  you have any) is required in the acknowledgements page and as the
%  last paragraph of text at the end of each respective chapter. See
%  the OGS Formatting Manual for more information.
%
\begin{acknowledgements}
\par A PhD is an important recognition from the scientific community and institution signifying the success and qualifications of an individual to pursue creative and original research.
This success is not spontaneous but typically builds upon many years of academic and other preparation.
I thank all of the innumerable mentors who have influenced me throughout the years.

\par I would like to acknowledge my advisors Rommie Amaro and Andy McCammon.
They have created a stimulating and rich yet flexible research environment which has afforded me the ability to explore my personal research interests with their full support.
Through the highs and lows of our work, they have guided me and pushed me to grow in all aspects of my life.
Their constant encouragement has allowed me to flourish in ways I never thought possible.
During my graduate tenure, perhaps most importantly, they have entrusted me with the role of mentoring many students across all levels of academic preparation.
They have enabled my passion for sharing the pursuit of science with others.
I will treasure this gift, thank you.

\par Rob Swift was a close mentor during the first two years of my degree.
He helped lay the foundations for my graduate success.
Rob, I have enjoyed watching your family grow.
Thanks for sharing your humor, scientific rigor, and skepticism which have forged me into the scientist I am today.
I also thank my committee members Katja Lindenberg, Mike Holst, Francesco Paesani, and Mike Gilson.
You were tough but fair, thanks for always pushing me to think critically.

\par Special thanks to all of the other current and former members of the Amaro and McCammon groups.
I have realized that listing all of your many contributions to my life and work could take me an additional 5 years.

\par \cref{chap:exascale}, in full, is a modified reprint of the material as it appears in ``Lee, C. T.; Amaro, R. E. \emph{Exascale Computing: A New Dawn for Computational Biology.} Comput. Sci. Eng. 2018, 20 (5), 18–25.''.
The dissertation author was the primary investigator and author of this work.

\par \cref{chap:permeability}, in full, is a modified reprint of the material as it appears in ``Lee, C. T.; Comer, J.; Herndon, C.; Leung, N.; Pavlova, A.; Swift, R. V.; Tung, C.; Rowley, C. N.; Amaro, R. E.; Chipot, C.; et al. \emph{Simulation-Based Approaches for Determining Membrane Permeability of Small Compounds.} J. Chem. Inf. Model. 2016, 56 (4), 721–733.''
The dissertation author was the primary investigator and author of this work.

\par \cref{chap:mileperm}, in full, is a modified reprint of the material as it appears in ``Votapka, L. W.$^{\dagger}$; Lee, C. T.$^{\dagger}$; Amaro, R. E. \emph{Two Relations to Estimate Membrane Permeability Using Milestoning.} J. Phys. Chem. B 2016, 120 (33), 8606–8616.''
The dissertation author was a primary coinvestigator and author of this work.

\par \cref{chap:bcd}, in full, is a modified reprint of the material as it appears in ``Jagger, B. R.; Lee, C. T.; Amaro, R. E. \emph{Quantitative Ranking of Ligand Binding Kinetics with a Multiscale Milestoning Simulation Approach.} J. Phys. Chem. Lett. 2018, 9 (17), 4941–4948.''
The dissertation author was an investigator, supervisor, and coauthor of this work.

\par \cref{chap:allostery}, in full, is a modified reprint of the material as it appears in ``Wagner, J. R.$^{\dagger}$; Lee, C. T.$^{\dagger}$; Durrant, J. D.; Malmstrom, R. D.; Feher, V. A.; Amaro, R. E. \emph{Emerging Computational Methods for the Rational Discovery of Allosteric Drugs.} Chem. Rev. 2016, 116 (11), 6370–6390.''.
The dissertation author was a primary coinvestigator and author of this work.

\par \cref{chap:ccr2}, in full, has been submitted for publication and is presented as it may appear in ``Taylor, B. C.; Lee, C. T.; Amaro, R. E. \emph{Structural Basis for Ligand Modulation of the CCR2 Conformational Landscape.} PNAS.''
The dissertation author was an investigator, supervisor, and coauthor of this work.

\par \cref{chap:asc}, in full, has been submitted for publication and is presented as it may appear in
``Lee, C. T.$^{\dagger}$; Moody, J. B.$^{\dagger}$; Amaro, R. E.; McCammon, J. A.; Holst, M. \emph{The Implementation of the Colored Abstract Simplicial Complex and Its Application to Mesh Generation.} Trans. Math. Soft.''
The dissertation author was a primary coinvestigator and author of this work.
\end{acknowledgements}


%% VITA
%
%  A brief vita is required in a doctoral thesis. See the OGS
%  Formatting Manual for more information.
%
\begin{vitapage}
\begin{vita}
  \item[2010-2011] Undergraduate Research Fellow\\ Shirts Lab, Department of Chemical Engineering, University of Virginia
  \item[2011] Bachelor of Science in Chemistry\\University of Virginia
  \item[2011] Bachelors of Arts in Computer Science\\
  University of Virginia
  \item[2011] Graduate Research Fellow\\
  Columbus and Mura Labs, Department of Chemistry, University of Virginia
  \item[2013] Master of Chemistry With a Specialization in Biochemistry\\ University of Virginia
  \item[2013-2019] Graduate Research Fellow\\ Amaro and McCammon Labs, University of California, San Diego
  \item[2019] Doctor of Philosophy in Chemistry\\ University of California, San Diego
\end{vita}
\begin{publications}
  \item Malmstrom, R. D.; \textbf{Lee, C. T.}; Van Wart, A. T.; Amaro, R. E. \emph{Application of Molecular-Dynamics Based Markov State Models to Functional Proteins.} J. Chem. Theory Comput. 2014, 10 (7), 2648–2657.
  \item Gray, C.; Price, C. W.; \textbf{Lee, C. T.}; Dewald, A. H.; Cline, M. A.; McAnany, C. E.; Columbus, L.; Mura, C. \emph{Known Structure, Unknown Function: An Inquiry-Based Undergraduate Biochemistry Laboratory Course.} Biochem. Mol. Biol. Educ. 2015, 43 (4), 245–262.
  \item Wagner, J. R.$^{\dagger}$; \textbf{Lee, C. T.$^{\dagger}$}; Durrant, J. D.; Malmstrom, R. D.; Feher, V. A.; Amaro, R. E. \emph{Emerging Computational Methods for the Rational Discovery of Allosteric Drugs.} Chem. Rev. 2016, 116 (11), 6370–6390.
  \item Votapka, L. W.$^{\dagger}$; \textbf{Lee, C. T.$^{\dagger}$}; Amaro, R. E. \emph{Two Relations to Estimate Membrane Permeability Using Milestoning.} J. Phys. Chem. B 2016, 120 (33), 8606–8616.
  \item \textbf{Lee, C. T.}; Comer, J.; Herndon, C.; Leung, N.; Pavlova, A.; Swift, R. V.; Tung, C.; Rowley, C. N.; Amaro, R. E.; Chipot, C.; et al. \emph{Simulation-Based Approaches for Determining Membrane Permeability of Small Compounds.} J. Chem. Inf. Model. 2016, 56 (4), 721–733.
  \item \textbf{Lee, C. T.}; Amaro, R. E. \emph{Exascale Computing: A New Dawn for Computational Biology.} Comput. Sci. Eng. 2018, 20 (5), 18–25.
  \item Jagger, B. R.; \textbf{Lee, C. T.}; Amaro, R. E. \emph{Quantitative Ranking of Ligand Binding Kinetics with a Multiscale Milestoning Simulation Approach.} J. Phys. Chem. Lett. 2018, 9 (17), 4941–4948.
  \item Taylor, B. C.; \textbf{Lee, C. T.}; Amaro, R. E. \emph{Structural Basis for Ligand Modulation of the CCR2 Conformational Landscape.} bioRxiv 2018, 392068. \textit{Submitted}
  \item \textbf{Lee, C. T.$^{\dagger}$}; Moody, J. B.$^{\dagger}$; Amaro, R. E.; McCammon, J. A.; Holst, M. \emph{The Implementation of the Colored Abstract Simplicial Complex and Its Application to Mesh Generation.} 2018. arXiv: 1807.01417. \textit{Submitted}
  \item \textbf{Lee, C. T.$^{\dagger}$}; Laughlin, J. G.$^{\dagger}$, Angliviel de La Beaumelle, N.; Amaro, R. E.; McCammon, J. A.; Ramamoorthi, R.; Holst, M. J.; Rangamani, P. \emph{GAMer 2: A System for 3D Mesh Processing of Cellular Electron Micrographs.} bioRxiv 2019. \textit{Submitted}
\end{publications}
\end{vitapage}

%% ABSTRACT
%
%  Doctoral dissertation abstracts should not exceed 350 words.
%   The abstract may continue to a second page if necessary.
%
\begin{abstract}
  The dream of any biologist is to have an Assimovian nanosubmarime which enables the direct viewing of how molecules interact.
  In the absence of this, we turn to...

  In this thesis, I describe my efforts in computing important physical properties such as passive membrane permeability of drug molecules, and binding/unbinding kinetics.
  I also
\end{abstract}
\end{frontmatter}
