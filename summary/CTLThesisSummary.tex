\documentclass[12pt]{article}

\usepackage[margin=1in]{geometry}
\usepackage{graphicx}
\usepackage{xspace}
\usepackage{xcolor}
\newcommand{\Red}[1]{{\color{red}{#1}}}
\usepackage{fontspec}
\setmainfont{Times New Roman}

\setlength{\parindent}{0.5\parindent}

\usepackage{titlesec}
\titleformat{\section}[runin]
  {\normalfont\normalsize\bfseries}{\thesection}{0.5em}{}
\titleformat{\subsection}[runin]
  {\normalfont\normalsize\bfseries}{\thesubsection}{0.5em}{}
\titleformat{\subsubsection}[runin]
  {\normalfont\normalsize\bfseries}{\thesubsubsection}{0.5em}{}
% \titleformat{\subparagraph}[runin]
  % {\normalfont\normalsize}{\underline{\thesubparagraph.\enspace}}{0.5em}{\underline}
\setcounter{secnumdepth}{6}

\titlespacing{\section}{0pt}{0.15em}{\parindent}
\titlespacing{\subsection}{0pt}{0.15em}{\parindent}
\titlespacing{\subsubsection}{0pt}{0.1em}{\wordsep}
\titlespacing{\paragraph}{0pt}{0em}{\wordsep}
\titlespacing{\subparagraph}{0pt}{0em}{\wordsep}


\begin{document}
% Times New Roman, 12 point font, 1 inch margins all around, 2 pages max
% Briefly summarize the intended research training, past training experiences, and career goals
\section{Summary}
In his thesis entitled ``Forging Pathways to Enable Multiscale Modeling of Cellular Scale Phenomena'', Dr. Christopher Lee presents an impressive body of work not just in quantity, but also quality and breadth.
His dissertation work spans topics ranging from the molecular modeling of drug permeability to mesh generation.
The quality and impact of his work is evident in the many local and national fellowships and awards he has received over the years.
A summary of his dissertation research, categorized into three major directions, and its impact is highlighted below.

\section{Computation of values useful to drug discovery}
The development of new drugs and therapeutics is an expensive endeavor.
Candidate compounds can fail at various stages in the drug discovery pipeline.
One popular strategy to reduce drug attrition and development cost is to employ \textit{in silico} methods to predict values of interest.
By coupling methods in statistics, physical chemistry, and computation, Chris has worked on the prediction of two fundamental rates important to drug discovery: passive membrane permeability, and drug binding kinetics.
Both of these values can be related to a candidate drug's dose-response and efficacy.
Briefly, a drug which cannot reach the area of effect due to poor permeability will fail.
Similarly, modern pharmacology suggests that drugs which are more consistently bound will have greater efficacy.
In ``Lee, C. T. et al. J. Chem. Inf. Model. 2016'', we assess the efficiency and accuracy of several methods for computing permeability using atomistic simulations.
Later in ``Votapka, L. W.*; Lee, C. T.*; Amaro, R. E. J. Phys. Chem. B 2016'', we describe and validate a new strategy to compute membrane permeability values using an enhanced sampling technique called milestoning.
Finally, we use the theory of milestoning in ``Jagger, B. R.; Lee, C. T.; Amaro, R. E. J. Phys. Chem. Lett. 2018'' to compute and rank order the binding kinetics of a host-guest system.

\subsection{Impact}
The cumulative impacts of these works are summarized as follows.
The prediction of permeability using atomistic simulations and enhanced sampling methods can suffer from the relaxation of orthogonal degrees of freedom, confounding predictions.
To address this, we suggest several strategies ranging from improved monitoring of lipid relaxation to an interpolative method to reduce the computational burden of future calculations.
We also present a new mathematical scheme to compute permeability using an unbiased enhanced sampling method called milestoning.
Since milestoning employs fewer assumptions, this work has the promise of improving the accuracy of future permeability calculations.
The applications of milestoning to compute drug binding kinetics is the first application of our methods to rank order compounds based on kinetics.
We are currently exploring a collaboration with a large pharmaceutical company to integrate the milestoning technology into their drug discovery pipeline.

\section{MSM Modeling and Allostery}
Traditional structure based rational drug design considers the binding of ligands to static protein structures as revealed by NMR or X-ray crystallography.
However, in vivo, proteins are constantly undergoing dynamic conformational changes.
New efforts in computer aided drug discovery seek to capitalize upon the dynamics of target proteins to identify so-called cryptic pockets (i.e., pockets not present in a static structure which open due to protein movement), and can uncover allosteric sites (i.e., binding regions located away from but dynamically coupled to the endogenous ligand pocket).
Discovered cryptic and allosteric pockets can reveal new opportunities for druggable sites.
We review the state-of-the-art for rationally designing allosteric drugs in ``Wagner, J. R.*; Lee, C. T.*; et al. Chem. Rev. 2016''.
In addition to the methods highlighted in this review, there is a pressing need for frameworks to quantitatively describe the dynamics of proteins.
In ``Malmstrom, R. D.; Lee, C.T.; et al. J. Chem. Theory Comput. 2014'', we discuss the practical application of Markov state models (MSM) to study biomolecular dynamics.
Drawing upon this background, we applied MSMs to study the dynamics of CCR2 in ``Taylor, B. C.; Lee, C. T.; Amaro, R. E. PNAS 2019''.
CCR2 is a class 2 GPCR which is correlated with a number of inflammatory diseases ranging from diabetes to cancer.
Historical efforts to develop CCR2 antagonists have failed in clinical trials.
In this study, we use MSMs to quantitatively decompose the drug bound and unbound dynamics to gain insights into how the drugs work and why they are failing.

\subsection{Impact}
Methods to discover and quantify new druggable pockets create new opportunities for pharmaceutical development.
Our review surveys many of the primary computational methods for studying allostery in protein systems.
It also discusses the benefits of potential allosteric drugs as a call to the general community.
The application of MSMs to create human interpretable representations from high dimensioned protein dynamics data has provided a quantitative framework for studying both protein kinase A and CCR2.
Of particular note, the work on CCR2 has revealed a new `druggable' cryptic pocket which opens to and extends beyond the known allosteric binding site.
This research has also yielded hints as to why previous antagonists have failed.
Moreover, with some additional mathematical study, we anticipate that the CCR2 simulations will serve as an ideal dataset for developing a comparative MSM framework which is capable of semiautomated and quantitative characterization of the effect of drug binding on protein dynamics.
We believe that such a framework, if successful, would enable the automation of molecular dynamics simulations and analysis for drug discovery.

\section{Mesh generation to enable image-based cellular models}
Recent advances in volume electron microscopy (EM) have, for the first time, enabled imaging of single cells in 3D at the nanometer length scale resolution.
Although these emerging datasets enable image correlation based structure comparison, an uncharted frontier is the ability to simulate cellular processes to query the impact of subcellular structure on signaling dynamics.
To achieve this goal, we will require a system for going from EM images to 3D volume meshes which can be used in finite element simulations.
In ``Lee, C. T.*; Laughlin, J. G.*, et al. bioRxiv 2019'', we describe the development of an end-to-end pipeline for this task by adapting and extending computer graphics mesh processing and smoothing algorithms.
This pipeline is applied to a series of electron micrographs of dendrite morphology explored at several length scales and we demonstrate that the meshes are suitable for finite element simulations.
This workflow makes use of our recently rewritten mesh processing software GAMer 2 which is built upon the CASC data structure developed and described in ``Lee, C. T.*; Moody, J. B.*; et al. ACM Trans. Math. Soft. In Press''.
Both the code for GAMer and CASC data structure are open source and available for download on GitHub.

\subsection{Impact:}
We posit that a new frontier at the intersection of computational technologies and single cell biology is now open.
Our innovations in algorithms to reconstruct and simulate cellular length scale phenomena based on emerging structural data will enable realistic physical models and advance discovery.
This work is a natural extension and improvement upon prior work in the field which has largely neglected geometry or employed idealized and unrealistic shapes.
We anticipate that physical modeling in realistic geometries will enable the query of subcellular structure-function relationships in the short term.
As these physical models mature, they will enable the prediction of emergent properties which may be useful for drug discovery along with basic science.
\end{document}
