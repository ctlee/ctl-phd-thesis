%!TEX ROOT = ../ctl-phd-thesis.tex

\par The goal of biology is to understand life to an extent that one can predict the eventual behaviors of a living system under or in the absence of a stimulus or perturbation.
Although many debate the utility of quantitative modeling in answering this question, I will abstain from such discussion and focus on the benefits of modeling as I view them.
The development of a whole cell or organism model has become a hot topic in modern biology\cite{Roberts2014}.
As biologists uncover the mysteries which make life possible, they unveil new complexities\cite{CheckHayden2010}.
These complex and non-linear behaviors are not amenable to scrutiny via traditional research methods which are driven by classical hypotheses (i.e., by varying one parameter at a time and observing and characterizing the results to form conclusions).
Simple experimental models, while elegant, test isolated model behaviors in simplified systems which are often not generalizable.
In order to derive a general conclusion, many expensive experiments must be run to explore the solution space at large potential cost.

\par One strategy to reduce the experimental cost burden is to shift towards inexpensive mathematical modeling.
Mathematical models in biology are increasing in popularity, and come in a wide variety of flavors\cite{Gunawardena2014}.
For example, there are information and data driven models as well as models based on the fundamentals of physical principles.
In this dissertation, I will focus on the development of methods in support of the latter.
By codifying our knowledge of the physical world into a model, we can gain many benefits.
First, physical models can serve as a glue to reconcile the results of individual experiments under isolated conditions.
Cleverly designed mathematical models are also amenable to computed solutions using numerical methods.
Combining knowledge and computation, inexpensive parameter sweeps that eliminate physical impossibilities and guide experimental efforts can serve to reduce the experimental cost of biology.

\par Physical models can also produce other benefits for science.
During the design and validation phase of model construction, when a physical model fails to reproduce experimental results, this is a litmus test for the limitations of our current understanding.
By studying the assumptions of the model and querying the differences between prediction and experiment, scientists can pinpoint possible knowledge gaps.
Once a physical model is sufficiently validated, due to the fundamental basis of transferable physical properties, the models can also be used to generate prospective predictions.
For example, in an inspiring study, Hodgkin and Huxley created a famous model describing the propagation of an action potential along a giant squid axon\cite{HUXLEY1952}.
Although the model was based on empirical fitting of electrophysiology data, many speculated upon the physical implications of the mathematics.
One term in particular, predicting the presence of four voltage-sensitive gates in the sodium ion channel, was later experimentally confirmed by crystallographic structure which resolved the tetrameric channel structure\cite{Sigg2014a}.

\par We live in an exciting time, where there is a convergence of emerging structural data, legacy experimental parameters, and computational power.
In order to understand life, we must embrace the complexity of biology.
The work described in this dissertation seeks to address several challenges which hinder the routine application of large systems biology models.
This includes the computational prediction of various biological rates such as diffusion, permeability, and binding/unbinding, strategies to develop models of individual protein dynamics, and the capacity to generate multiscale geometric models suitable for biological modeling.

\section{Overview of Chapter Contents}

\par To reiterate and highlight the opportune convergence of computing and biological data, \cref{chap:exascale} discusses the potential applications and gains afforded by future exascale supercomputers to biology.
This chapter provides some perspective on the role that computers can play to solve various biological problems.
I speculate upon not only the contributions of computation to basic science, but also upon the impact of computation and technology on translational fields such as personalized medicine.
In this chapter, I also address and frame some of the aforementioned challenges to future modeling efforts with solutions described in the subsequent chapters.

\par Owing to the complexity of biology, a complete systems model will require the definition of many rates.
Due to the basic nature of many of these measurements, the experimental work to measure these values may not be publishable and is therefore difficult to source.
Although robotics which automate assay development and screening are amenable to some tasks, the complexity of experimental setups for measuring kinetic properties can often hinder accurate and high-throughput measurement.
To augment the existing data and measurements, molecular simulations such as Molecular Dynamics (MD) can be employed to generate estimates\cite{Leach2001,Durrant2011a}.
MD effectively integrates the classical equations of motion for an atomistic system.
Given that the typical timescales of simulation are typically much shorter than the timescales of the process of interest, brute force MD simulations can rarely capture the dynamics of interest with sufficient statistics.
Instead, using strategies derived from statistical mechanics, many enhanced sampling strategies have been developed to facilitate the estimation of classically intractable properties\cite{Chipot2007,Tuckerman2010}.
In \cref{chap:permeability}, I compare the efficiency of four enhanced sampling strategies to compute drug passive membrane permeability.
We find that there are many orthogonal degrees of freedom which can hinder the convergence of these calculations and recommend some best practices for future work.

\par While studying membrane permeability, I had many stimulating discussions with my colleague Lane Votapka on the topic.
During our overlap in the group, Lane was working on the development of a software tool which facilitates the use of milestoning to compute various rates.
Milestoning is another enhanced sampling strategy where, instead of observing the full transition from start to finish of a process, we collect statistics of transitions along many small segments of the full pathway and stitch them together in postprocessing\cite{Faradjian2004,Majek2010,Vanden-Eijnden2008,Kirmizialtin2011,Votapka2015,Bello-Rivas2015,Votapka2017c}.
The application of milestoning theory to permeability calculations was not straightforward since the natural currency of milestoning is transition probabilities, while the expressions to compute permeability required potentials of mean force.
To solve this, Lane derived a new relation to compute permeability values using transition probabilities from milestoning which we validated using a toy model inspired by my prior work on permeability.
This research is described in \cref{chap:mileperm}.

\par After learning about the virtues of the milestoning method, I started thinking about possible applications of milestoning to compute other rates of interest.
By chance, my colleague Ben Jagger and I met Chia-en Chang at a local conference where she was presenting preliminary results describing brute-force MD calculations of host-guest binding kinetics and thermodynamics\cite{Tang2017}.
This dataset which contained several ligands, experimentally determined kinetics, and brute-force MD was the ideal system to benchmark the use of SEEKR\cite{Votapka2017c}.
Chia-en kindly shared the dataset with us and working with Ben, we predicted the binding kinetics of the system using milestoning.
The results of this work are described in \cref{chap:bcd} along with some best practices to monitor the convergence of computed rate estimates which we developed along the way.

\par Also of interest for biological simulations is the dynamics of protein movement.
Not only are the dynamics of proteins critical to their biological function, but protein dynamics also influence phenomena such as drug binding.
In \cref{chap:allostery}, I present a review of many different strategies to use computers to inform the rational design of allosteric drugs.
Previously I had worked with Robert Malmstrom on a paper describing the applications of Markov state models to model protein kinase A dynamics with and without cyclic AMP\cite{Malmstrom2015a}.
During this project, we started thinking about the use of MSMs to model perturbations to the conformational ensembles of protein targets by drugs and using the information to support drug discovery efforts.
Later working with Bryn Taylor, we investigated the impact of drug binding to GPCR CCR2.
The generation and description of our CCR2 MSMs is presented in \cref{chap:ccr2}.

\par Another important aspect of cell modeling is the generation of computable geometries to represent cellular scenes.
Meshes are commonly used in engineering fields as a general geometric representation.
The use of meshes is desirable since they are often also compatible with simulation methods such as finite elements.
However, the numerical behavior of algorithms is often limited by mesh conditioning.
To support the development of robust mesh generation codes for biological geometries, working with John Moody, I developed the Colored Abstract Simplicial Complex library which is described in \cref{chap:asc}.

\par In summary, the work in this dissertation explores the use of various modeling strategies to predict values of interest or to generate geometric meshes in support of future multiscale modeling efforts.

