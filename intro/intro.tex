%!TEX ROOT = ../ctl-phd-thesis.tex

\par The goal of biology is to understand life such that one can predict the eventual behaviors of a living system.
Although many debate the utility of quantitative modeling in biology and chemistry, I will abstain from such discussion and focus on the benefits of such exercises as I view them.
To understand life such that one can predict the eventual behavior or a living system.


\cite{Gunawardena2014}
Elimination of physical impossibilities which can reduce the costs of expensive wet-lab work.
Codifying our understanding and comparing with observations can give us an idea of whether or not the model is complete.
Are there certain conditions which are not well represented?
Where the model fails to produce meaningful results, tells us where our understanding is lacking.
Well validated models can also generate new predictions to inspire new experiments.

\par When an experimental setup is difficult or contains great peril, models can provide a safer way to study phenomena of interest.


\par Mathematical modeling Hodgkin-Huxley model predicting sodium ion channel



There are many kinds of models ranging from phsyical to

\par Convergence of structural data, experimental parameters and computation.

\section{Overview of Chapter Contents}

\par

