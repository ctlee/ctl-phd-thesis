%!TEX ROOT = ../ctl-phd-thesis.tex

\par The goal of biology is to understand life to an extent that one can predict the eventual behaviors of a living system under or in the absence of a stimulus or perturbation.
Although many debate the utility of quantitative modeling in answering this question, I will abstain from such discussion and focus on the benefits of modeling as I view them.
The development of a whole cell or organism model, is a hot topic in modern biology.
Technological advances in experimental methods, have lead to an unprecendented and detailed view into various components of biology.
Biochemists can design and execute experiments driven by classical hypothesis, i.e., by varying one parameter at a time and observing and characterizing the results to form conclusions.
However, biology is complex and non-linear.
Simple experimental models, while elegant, often test isolated model behaviors in simplified systems which are not representative of other experimental conditions.
Many expensive experiments can be run to explore the solution spaces at large potential cost.
% To fully understand biology, one must embrace the complexity present in real biological systems.

\par One strategy to reduce this cost burden is to turn to relatively inexpensive modeling.
Mathematical models in biology are increasing in popularity, and come in a wide variety of flavors\cite{Gunawardena2014}.
For example, there are information and data driven models as well as models based on the fundamentals of physical principles.
In this dissertation, I will focus on the development of methods in support of the latter.
By codifying our knowledge into a physical model, we can gain many benefits.
Physical models can serve as a glue to reconcile the results of individual experiments at isolated conditions.
Given sufficient validation, these models can be used to generate prospective predictions to inspire new experiments.
When a physical model fails to reproduce experimental results, this is a litmus test for the limitations of our current understanding.


Models can also be used to explore the unknown.
Owing to the reduced cost, parameter sweeps in model space can be performed to eliminate physical impossibilities to guide future experimental efforts.


We are at an exciting time, where there is a convergence of structural data, experimental parameters and computation.

\par Mathematical modeling Hodgkin-Huxley model predicting sodium ion channel\cite{Sigg2014a}


There are many kinds of models ranging from physical to

\par

\section{Overview of Chapter Contents}

\par \cref{chap:exascale} discusses the potential applications and gains afforded by future exascale supercomputers.

