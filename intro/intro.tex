%!TEX ROOT = ../ctl-phd-thesis.tex

\par The goal of biology is to understand life to an extent that one can predict the eventual behaviors of a living system under or in the absence of a stimulus or perturbation.
Although many debate the utility of quantitative modeling in answering this question, I will abstain from such discussion and focus on the benefits of modeling as I view them.
The development of a whole cell or organism model, is a hot topic in modern biology.
Technological advances in experimental methods, have lead to an unprecendented and detailed view into the complexity of biology.
Biochemists design and execute elegant experiments driven by classical hypothesis, varying one parameter at a time and observing and characterizing the results to form conclusions.
However, biology is complex and non-linear.

\par Modeling can be the glue to reconcile the results of individual experiments under isolated conditions, to produce a greater understanding of a system of interest.
We are at an exciting time, where there is a convergence of structural data, experimental parameters and computation.

\cite{Karr2012}
% CITE B. Palsson

By encoding our understanding into a mathematical form, we get many benefits.
\cite{Gunawardena2014}

Elimination of physical impossibilities which can reduce the costs of expensive wet-lab work.
Codifying our understanding and comparing with observations can give us an idea of whether or not the model is complete.
Are there certain conditions which are not well represented?
Where the model fails to produce meaningful results, tells us where our understanding is lacking.
Well validated models can also generate new predictions to inspire new experiments.

\par When an experimental setup is difficult or contains great peril, models can provide a safer way to study phenomena of interest.


\par Mathematical modeling Hodgkin-Huxley model predicting sodium ion channel\cite{Sigg2014a}


There are many kinds of models ranging from physical to

\par

\section{Overview of Chapter Contents}

\par \cref{chap:exascale} discusses the potential applications and gains afforded by future exascale supercomputers.

